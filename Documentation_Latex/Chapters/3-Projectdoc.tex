%*******10********20********30********40********50********60********70********80

%For all chapters, use the newdefined chap{} instead of chapter{}
% This will make the text at the top-left of the page be the same as the chapter

\chap{Project Documentation}

\section{Project organization and testing}
As a starting point, the main Scene and Controller handling server selection and user login, as well as the functionality to set and switch between scenes was implemented. \\
On the top of these fundamental construct, the login screen was adapted from the previous exercises was adapted and the rest of the application function was split between the 3 main screens ("Vocabulary", "Learning", and "Goals") and style and design was added by extending the .css file and adding icons, designs etc. \\
In each developmental area a GUI implementing the required features was designed and the required internal data-structures, interfaces between screens, and API communication were determined and implemented as needed. \\
The generated code was manually tested by the specified use-cases.

\section{Project Responsibilities and Contributions}

The project was split amongst the frames within in the application. Responsibilities were distributed as following:

\begin{itemize}
    \item Bela Bothin was responsible for %  ~53 punkte
    \begin{itemize}
        \item Setting up \textit{Main} class and providing function to change scenes
        \item Registration frame: Login and register screen (adapted from previous exercise assignments), server selection and connection, waiting animation, Remember-me function, and password encryption
        \item Application design: Usage of css file and addition of the icons, with the structure and design of MainScene containing the NavigationBar and TitleBar
        \item API calls (used for Vocabulary learning and management)
    \end{itemize}
\item Maximilian Engels was responsible for % ~27 punkte
    \begin{itemize}
        \item Vocabulary learning: Core functionality\\
        and string operations like checking the users answer for mistakes
    \end{itemize}
\item Lukas Rademacher was responsible for % ~33 punkte
    \begin{itemize}
        \item Vocabulary management: Accessing, adding, searching, editing, and deleting Vocabulary entries
    \end{itemize}
\item Sebastian Rassmann was responsible for: % ~21 punkte 
    \begin{itemize}
        \item Setting up Gradle project and gitignore
        \item Selection of vocabulary to learn ("Goals"): Selecting vocabulary to learn and filter by language - the code for the vocabulary \textit{TableView} with check-boxes (\textit{VocabSelection} class and state accessing logic) was re-used for vocabulary management
        \item Documentation
    \end{itemize}
\end{itemize}

Usage of FXML and the respective controllers, internationalization, and asynchronous API calls were realized by each developer individually. \\
Remaining jointly used functionality, especially, the \textit{Variables} class used for inter-screen communication was developed jointly. 

\section{Vocabulary List}
The Vocabulary list pane shows up, after the user has logged in to the application.
It consists out of a table with all the vocables and the functions search/filter, edit-mode, edit, add and delete.\\ More information and how they were created are following in the next chapters. Furthermore these chapters represent a vague timeline of how things have changed from version to version as well as a rough description of how each function was implemented. (When there is no section titled after a specific function, nothing has been changed in this relating version.) 